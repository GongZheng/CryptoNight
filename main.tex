\documentclass{article}
\usepackage[utf8]{inputenc}

\title{Boost CryptoNight}
\author{Guohong Liao, Zhijie Liu, Peiran Luo, Zheng Gong}
\date{}

\begin{document}

\maketitle

\section{Introduction}
In blockchain applications, the proof-of-work (PoW) mechanism is pivotal for realizing its design rationale. A good PoW mechanism should be equal for distributed miners with different computation powers, such a blockchain system that is build on this PoW can be uncenterized and robust. CryptoNight is a memory-hard hash function. It is designed to be inefficiently computable on GPU, FPGA and ASIC architectures. The CryptoNight algorithm's first step is initializing large scratchpad with pseudo-random data. The next step is numerous read/write operations at pseudo-random addresses contained in the scratchpad. The final step is hashing the entire scratchpad to produce the resulting value.

\section{Preliminaries}
\definition{hash function.} An efficiently computable function which maps data of
   arbitrary size to data of fixed size and behaves similarly to a
   random function.

\defintion{scratchpad.} A large area of memory used to store intermediate values
   during the evaluation of a memory-hard function

\section{The CryptoNight hash function}
\subsection{The design rationale}

\subsection{Scratchpad Initialization}
First, the input is hashed using Keccak [KECCAK] with parameters b =
   1600 and c = 512. The bytes 0..31 of the Keccak final state are
   interpreted as an AES-256 key [AES] and expanded to 10 round keys. A
   scratchpad of 2097152 bytes (2 MiB) is allocated. The bytes 64..191
   are extracted from the Keccak final state and split into 8 blocks of
   16 bytes each. Each block is encrypted using the following procedure:

      for i = 0..9 do:
          block = aes_round(block, round_keys[i])

   Where aes_round function performs a round of AES encryption, which
   means that SubBytes, ShiftRows and MixColumns steps are performed on
   the block, and the result is XORed with the round key. Note that
   unlike in the AES encryption algorithm, the first and the last rounds
   are not special. The resulting blocks are written into the first 128
   bytes of the scratchpad. Then, these blocks are encrypted again in
   the same way, and the result is written into the second 128 bytes of
   the scratchpad. Each time 128 bytes are written, they represent the
   result of the encryption of the previously written 128 bytes. The
   process is repeated until the scratchpad is fully initialized.

   This diagram illustrates scratchpad initialization:

 \subsection{Memory-hard loop}
   Prior to the main loop, bytes 0..31 and 32..63 of the Keccak state
   are XORed, and the resulting 32 bytes are used to initialize
   variables a and b, 16 bytes each. These variables are used in the
   main loop. The main loop is iterated 524,288 times. When a 16-byte
   value needs to be converted into an address in the scratchpad, it is
   interpreted as a little-endian integer, and the 21 low-order bits are
   used as a byte index. However, the 4 low-order bits of the index are
   cleared to ensure the 16-byte alignment. The data is read from and
   written to the scratchpad in 16-byte blocks. Each iteration can be
   expressed with the following pseudo-code:

      scratchpad_address = to_scratchpad_address(a)
      scratchpad[scratchpad_address] = aes_round(scratchpad 
        [scratchpad_address], a)
      b, scratchpad[scratchpad_address] = scratchpad[scratchpad_address],
        b xor scratchpad[scratchpad_address]
      scratchpad_address = to_scratchpad_address(b)
      a = 8byte_add(a, 8byte_mul(b, scratchpad[scratchpad_address]))
      a, scratchpad[scratchpad_address] = a xor 
        scratchpad[scratchpad_address], a

   Where, the 8byte_add function represents each of the arguments as a
   pair of 64-bit little-endian values and adds them together,
   component-wise, modulo 2^64. The result is converted back into 16
   bytes.

   The 8byte_mul function, however, uses only the first 8 bytes of each
   argument, which are interpreted as unsigned 64-bit little-endian
   integers and multiplied together. The result is converted into 16
   bytes, and finally the two 8-byte halves of the result are swapped.


 \subsection{Final output}
 After the memory-hard part, bytes 32..63 from the Keccak state are
   expanded into 10 AES round keys in the same manner as in the first
   part.

   Bytes 64..191 are extracted from the Keccak state and XORed with the
   first 128 bytes of the scratchpad. Then the result is encrypted in
   the same manner as in the first part, but using the new keys. The
   result is XORed with the second 128 bytes from the scratchpad,
   encrypted again, and so on. 

   After XORing with the last 128 bytes of the scratchpad, the result is
   encrypted the last time, and then the bytes 64..191 in the Keccak
   state are replaced with the result. Then, the Keccak state is passed
   through Keccak-f (the Keccak permutation) with b = 1600. 

   Then, the 2 low-order bits of the first byte of the state are used to
   select a hash function: 0=BLAKE-256 [BLAKE], 1=Groestl-256 [GROESTL],
   2=JH-256 [JH], and 3=Skein-256 [SKEIN]. The chosen hash function is
   then applied to the Keccak state, and the resulting hash is the
   output of CryptoNight.

\section{Appendix}
The testvector for CryptoNight is listed as follows:
      Empty string:
      eb14e8a833fac6fe9a43b57b336789c46ffe93f2868452240720607b14387e11.

      "This is a test":
      a084f01d1437a09c6985401b60d43554ae105802c5f5d8a9b3253649c0be6605.

\end{document}

